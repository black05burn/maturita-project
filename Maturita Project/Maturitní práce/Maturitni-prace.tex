\documentclass[12pt]{article}
\usepackage[utf8]{inputenc}
\usepackage
[
        a4paper,% other options: a3paper, a5paper, etc
        left=3.5cm,
        right=2.5cm,
        top=2.5cm,
        bottom=2.5cm,
        % use vmargin=2cm to make vertical margins equal to 2cm.
        % us  hmargin=3cm to make horizontal margins equal to 3cm.
        % use margin=3cm to make all margins  equal to 3cm.
]
{geometry}

\usepackage{url}
%\usepackage{hyperref}

\usepackage{fancyhdr}
\pagestyle{fancy}
\fancyhf{}
\renewcommand{\headrulewidth}{0pt}

\usepackage[czech]{babel}

\lfoot{Maturitní práce 2017/2018 - E4.C - Beneš David}
\cfoot{}
\rfoot{\thepage}

\usepackage{biblatex}
\addbibresource{bibtex.bib}


\title{}
\author{}
\date{}

\begin{document}
%ÚVODNÍ STRÁNKA 
    \begin{titlepage}
        \begin{center}
            \textbf{GYMNÁZIUM, STŘEDNÍ PEDAGOGICKÁ ŠKOLA, OBCHODNÍ
            AKADEMIE A JAZYKOVÁ ŠKOLA s právem státní jazykové zkoušky
            ZNOJMO, příspěvková organizace} 
            \vspace{2mm}
            \hline
             \vspace{2mm}
            \tiny{669 02  Znojmo, Pontassievská 3 • tel.: 515158101 • fax: 515225230 • e-mail: info@gpoa.cz • www.gpoa.cz, IČ:49438816}
        \end{center}
            \vspace{1.5cm}
        \begin{center}
            \textbf{\large{Maturitní práce}}
        \end{center}
        \vspace{1.5cm}
        Žák: David Beneš \\
        Třída: E4.C \\
        Školní rok: 2017/2018\\
        Obor vzdělání: Informační technologie\\
        [1.5cm]
        Název maturitní práce: In Progress\\
        [1.5cm]
        Vedoucí maturitní práce: Mgr. Milička Tomáš\\
        Oponent maturitní práce: \\
        [1.5cm]
        Sociální partner: GYMNÁZIUM, STŘEDNÍ PEDAGOGICKÁ ŠKOLA, OBCHODNÍ 
        
        \hspace{2.5cm}AKADEMIE A JAZYKOVÁ ŠKOLA s právem státní jazykové
        
        \hspace{2.5cm}zkoušky ZNOJMO, příspěvková organizace\\
        [1.5cm]
        Práce odevzdána dne: (datum)\\
        [1.5cm]
        Podpis žáka: (podpis)

    \end{titlepage}
%PROHLÁŠNÍ 
    \section*{Prohlášení}
    \vspace{1.5cm}
    Prohlašuji,  že  jsem  maturitní  práci  vypracoval  samostatně  pod  vedením  Mgr. Tomáše Miličky  a  uvedl  v seznamu  literatury  a  zdrojů veškerou  použitou  literaturu  a  další informační zdroje.\\
    [1.5cm]
    Místo datum\\
    [2cm]
    David Beneš
    
    \pagebreak
%ANOTACE 
    \section*{Anotace}
    \pagebreak
%OBSAH
    \tableofcontents
    
    \pagebreak
%ÚVOD
    \section{Úvod}
        Téma jsem si vybral, protože mě vždy zajímalo, jak se hry vyvýjí. Dále jsem chtěl vědět, jestli vůbec takovou hru dokážu naprogramovat.
        \subsection{Vymezení cílů}
            Můj hlavní cíl byl naprogramovat hru pomocí jazyka C\#. Hra by měla být jednoduchá na naučení, ale těžká na pokoření.\cite{website:ez2play-hard2master} Hra by měla mít jednoduchý a přehledný game design. Hra by spíše se měla zaměrovat na schovávání se před nepřáteli, než jejich zabíjením.
    \pagebreak
    \section{Využité programy}
    \subsection{Unity}
        \subsubsection{Představení}
            Unity je herní engine. %citace na wiki asi
            Je používaný pro vývoj jak indie (nezávislých), tak AAA her. Je v něm možné naprogramovat hru v jazyce C\#, který má mnoho přidaných Unity knihoven, ale i v JavaScriptu.
        \subsubsection{Důvod použití}
            Unity jsem si vybral, protože se mi hned na první pohled zalíbilo. Má krásný a jednoduchý design. Také znám spoustu her, které jsou za pomocí Unity enginu vytvořené. %citace WIKI UNITY GAMES 
            Velikým důvodem pro výběr byla také dostupnost programování v jazyce C\#, což ne všechny herní enginy nabízejí.
            
        \subsubsection{Výhody}
            \begin{itemize}
                \item Jednoduchost
                \item Je zdarma pro osobní i komerční účely (do \$100k)
                \item Dostupnost jazyka C\#
            \end{itemize}
        \subsubsection{Nevýhody}
            \begin{itemize}
                \item Pro úplné nástroje je třeba zaplatit
            \end{itemize}
        \subsubsection{Alternativy}
            \begin{itemize}
                \item Unreal Engine
                    \subitem Od společnosti Epic Games
                    \subitem Více se zaměřuje na akce/reacke systém, než se zaměřuje na samotné programování.
                \item CryEngine
                    \subitem Od společnosti Crytech
            \end{itemize}
    
    
    \subsection{Visual Studio}
    \subsubsection{Představení}
        Pro upravování C\# skritpů jsem použil program Visual Studio Community 2017 \cite{website:vs}
    \subsubsection{Výhody}
        \begin{itemize}
                \item Je zdarma pro osobní i komerční účely pro jedince
                \item Dostupnost jazyka C\#
                \item Jednoduché debugování
            \end{itemize}
    \subsubsection{Nevýhody}
        \begin{itemize}
                \item Není multiplatformní (neexistuje verze pro linux)
            \end{itemize}
    
    \pagebreak
    
    \section{Průběh vývoje}
    \subsection{Výběr žánru}
        Výběr žánru nebyl jednoduchý. Ze začátku jsem nevěděl, co přesně od hry očekávám. Začal jsem přemýšlet, co by mě osobně bavilo. Přišel jsem na dungeon crawler, kde hráč prochází jakýmsi doupětem a neustálé opakování úrovně má na denním pořádku. \cite{website:wiki-dungeoncrawler}
    
    
    
%ODKAZY
    \nocite{*}
    \pagebreak
    \printbibliography
\end{document}

